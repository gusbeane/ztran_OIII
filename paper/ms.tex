% mnras_template.tex
%
% LaTeX template for creating an MNRAS paper
%
% v3.0 released 14 May 2015
% (version numbers match those of mnras.cls)
%
% Copyright (C) Royal Astronomical Society 2015
% Authors:
% Keith T. Smith (Royal Astronomical Society)

% Change log
%
% v3.0 May 2015
%    Renamed to match the new package name
%    Version number matches mnras.cls
%    A few minor tweaks to wording
% v1.0 September 2013
%    Beta testing only - never publicly released
%    First version: a simple (ish) template for creating an MNRAS paper

%%%%%%%%%%%%%%%%%%%%%%%%%%%%%%%%%%%%%%%%%%%%%%%%%%
% Basic setup. Most papers should leave these options alone.
\documentclass[a4paper,fleqn,usenatbib]{mnras}

% MNRAS is set in Times font. If you don't have this installed (most LaTeX
% installations will be fine) or prefer the old Computer Modern fonts, comment
% out the following line
\usepackage{newtxtext,newtxmath}
% Depending on your LaTeX fonts installation, you might get better results with one of these:
%\usepackage{mathptmx}
%\usepackage{txfonts}

% Use vector fonts, so it zooms properly in on-screen viewing software
% Don't change these lines unless you know what you are doing
\usepackage[T1]{fontenc}
\usepackage{ae,aecompl}


%%%%% AUTHORS - PLACE YOUR OWN PACKAGES HERE %%%%%

% Only include extra packages if you really need them. Common packages are:
\usepackage{graphicx}	% Including figure files
\usepackage{amsmath}	% Advanced maths commands
\usepackage{amssymb}	% Extra maths symbols

%%%%%%%%%%%%%%%%%%%%%%%%%%%%%%%%%%%%%%%%%%%%%%%%%%

%%%%% AUTHORS - PLACE YOUR OWN COMMANDS HERE %%%%%

\newcommand{\avg}[1]{\ensuremath{\langle #1 \rangle}}
\newcommand{\CII}{\ensuremath{\text{C~\textsc{ii}}}}
\newcommand{\OIII}{\ensuremath{\text{O~\textsc{iii}}}}

% Please keep new commands to a minimum, and use \newcommand not \def to avoid
% overwriting existing commands. Example:
%\newcommand{\pcm}{\,cm$^{-2}$}	% per cm-squared

%%%%%%%%%%%%%%%%%%%%%%%%%%%%%%%%%%%%%%%%%%%%%%%%%%

%%%%%%%%%%%%%%%%%%% TITLE PAGE %%%%%%%%%%%%%%%%%%%

% Title of the paper, and the short title which is used in the headers.
% Keep the title short and informative.
\title[The Transition Redshift]{The transition redshift as a probe of reionization}

% The list of authors, and the short list which is used in the headers.
% If you need two or more lines of authors, add an extra line using \newauthor
\author[A. Beane and K. Moriwaki]{
Angus Beane$^{1}$\thanks{E-mail: angus.beane@cfa.harvard.edu}
and Kana Moriwaki$^{2}$\thanks{E-mail: kana.moriwaki@utap.phy.s.u-tokyo.ac.jp}
\\
% List of institutions
$^{1}$Center for Astrophysics {\normalfont |} Harvard \& Smithsonian, 60 Garden Street, Cambridge, MA 02138, USA\\
$^{2}$Department of Physics, The University of Tokyo, 7-3-1 Hongo, Bunkyo, Tokyo 113-0033, Japan
% $^{2}$Center for Computational Astrophysics, Flatiron Institute, 162 5th Avenue, New York, NY 10010, USA\\
% $^{3}$Department of Physics \& Astronomy, University of Pennsylvania, 209 South 33rd Street, Philadelphia, PA 19104, USA
}

% These dates will be filled out by the publisher
\date{Accepted XXX. Received YYY; in original form ZZZ}

% Enter the current year, for the copyright statements etc.
\pubyear{2019}

% Don't change these lines
\begin{document}
\label{firstpage}
\pagerange{\pageref{firstpage}--\pageref{lastpage}}
\maketitle

% Abstract of the paper
\begin{abstract}
The early stages of the Epoch of Reionization (EoR), probed by the 21~cm line,
are sensitive to the detailed formation of the first galaxies. An important
transition occurs on large scales at $\avg{x_i}\sim0.15$, when the 21~cm field
transitions from being positively correlated with the galaxy field to being
negatively correlated. The redshift at which this transition occurs (the
``transition redshift'') is, in principle, sensitive to both the structure of
the 21~cm field but also the galaxy field. However, by considering a fixed
model for the EoR and varying the parameters of a simple model for the galaxy
field in a mock [O~\textsc{iii}]-emitter sample cross-correlation, we show
that the transition redshift is in fact insensitive to the structure of the
selected galaxy sample. We show that prospects for measuring the transition
redshift are feasible in the near-term with a combination of LOFAR and JWST.
Being derived from a cross-correlation, the transition redshift is less
impacted by foreground contamination currently inhibiting 21~cm auto-spectrum
measurements. Therefore, the transition redshift is a promising probe of the
early stages of the EoR.
\end{abstract}

% Select between one and six entries from the list of approved keywords.
% Don't make up new ones.
\begin{keywords}
cosmology: theory -- dark ages, reionization, first stars -- large-scale
structure of Universe
\end{keywords}

%%%%%%%%%%%%%%%%%%%%%%%%%%%%%%%%%%%%%%%%%%%%%%%%%%

%%%%%%%%%%%%%%%%% BODY OF PAPER %%%%%%%%%%%%%%%%%%

\section{Introduction}
The redshifted 21~cm line contains a great deal of information about the
structure and evolution of the Epoch of Reionization (EoR). Once measured at
$z\sim6\textup{--}12$, it should reveal the nature of the formation of the
first stars, galaxies, and black holes, as well as the large-scale structure
of the Universe at high redshift \citep{2013fgu..book.....L}. While
instruments have reached the sensitivity necessary to detect the signal during
the EoR, issues of foreground contamination and instrumental artifacts have so
far prevented the success of such efforts. Only upper limits have been placed
on the amplitude of the 21~cm fluctuations \citep[e.g.][]{2013MNRAS.433..639P,
2014PhRvD..89b3002D, 2016ApJ...833..102B, 2017ApJ...838...65P}.

\section{To cite}
Papers that need to be cited:

The Cross-Correlation of High-Redshift 21 cm and Galaxy Surveys
\url{https://arxiv.org/abs/astro-ph/0611274}

Probing Reionization with the 21 cm-Galaxy Cross Power Spectrum
\url{https://arxiv.org/abs/0806.1055}

The HI Bias during the Epoch of Reionization, discusses the positive/negative
correlation transition as a function of $k$
\url{https://arxiv.org/abs/1907.05098}

\section{Conclusions}

The conclusions.

\section*{Acknowledgements}

The Acknowledgements section.

%%%%%%%%%%%%%%%%%%%%%%%%%%%%%%%%%%%%%%%%%%%%%%%%%%

%%%%%%%%%%%%%%%%%%%% REFERENCES %%%%%%%%%%%%%%%%%%

% The best way to enter references is to use BibTeX:

%\bibliographystyle{mnras}
%\bibliography{example} % if your bibtex file is called example.bib

\appendix

\section{Some extra material}

The appendix.

%%%%%%%%%%%%%%%%%%%%%%%%%%%%%%%%%%%%%%%%%%%%%%%%%%


% Don't change these lines
\bsp	% typesetting comment
\label{lastpage}
\end{document}