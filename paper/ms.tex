% mnras_template.tex
%
% LaTeX template for creating an MNRAS paper
%
% v3.0 released 14 May 2015
% (version numbers match those of mnras.cls)
%
% Copyright (C) Royal Astronomical Society 2015
% Authors:
% Keith T. Smith (Royal Astronomical Society)

% Change log
%
% v3.0 May 2015
%    Renamed to match the new package name
%    Version number matches mnras.cls
%    A few minor tweaks to wording
% v1.0 September 2013
%    Beta testing only - never publicly released
%    First version: a simple (ish) template for creating an MNRAS paper

%%%%%%%%%%%%%%%%%%%%%%%%%%%%%%%%%%%%%%%%%%%%%%%%%%
% Basic setup. Most papers should leave these options alone.
\documentclass[a4paper,fleqn,usenatbib]{mnras}

% MNRAS is set in Times font. If you don't have this installed (most LaTeX
% installations will be fine) or prefer the old Computer Modern fonts, comment
% out the following line
\usepackage{newtxtext,newtxmath}
% Depending on your LaTeX fonts installation, you might get better results with one of these:
%\usepackage{mathptmx}
%\usepackage{txfonts}

% Use vector fonts, so it zooms properly in on-screen viewing software
% Don't change these lines unless you know what you are doing
\usepackage[T1]{fontenc}
\usepackage{ae,aecompl}


%%%%% AUTHORS - PLACE YOUR OWN PACKAGES HERE %%%%%

% Only include extra packages if you really need them. Common packages are:
\usepackage{graphicx}	% Including figure files
\usepackage{amsmath}	% Advanced maths commands
\usepackage{amssymb}	% Extra maths symbols

%%%%%%%%%%%%%%%%%%%%%%%%%%%%%%%%%%%%%%%%%%%%%%%%%%

%%%%% AUTHORS - PLACE YOUR OWN COMMANDS HERE %%%%%

\newcommand{\beq}{\begin{equation}}
\newcommand{\eeq}{\end{equation}}

\newcommand{\tfast}{\texttt{21cmFAST}}

\newcommand{\avg}[1]{\ensuremath{\langle #1 \rangle}}
\newcommand{\CII}{\ensuremath{\text{C~\textsc{ii}}}}
\newcommand{\HII}{\ensuremath{\text{H~\textsc{ii}}}}
\newcommand{\OIII}{\ensuremath{\text{O~\textsc{iii}}}}

\newcommand{\Msun}{\ensuremath{\text{M}_\odot}}
\newcommand{\kpc}{\ensuremath{\text{kpc}}}
\newcommand{\Mpc}{\ensuremath{\text{Mpc}}}
\newcommand{\Myr}{\ensuremath{\text{Myr}}}
\newcommand{\Gyr}{\ensuremath{\text{Gyr}}}
\newcommand{\kms}{\ensuremath{\text{km}\,\text{s}^{-1}}}

\newcommand{\hoverMpc}{\ensuremath{h\,\text{Mpc}^{-1}}}

\newcommand{\fid}{\texttt{fid}}
\newcommand{\hot}{\texttt{hot}}
\newcommand{\cold}{\texttt{cold}}

\newcommand{\zst}{\ensuremath{z_{\star}}}

% Please keep new commands to a minimum, and use \newcommand not \def to avoid
% overwriting existing commands. Example:
%\newcommand{\pcm}{\,cm$^{-2}$}	% per cm-squared

%%%%%%%%%%%%%%%%%%%%%%%%%%%%%%%%%%%%%%%%%%%%%%%%%%

%%%%%%%%%%%%%%%%%%% TITLE PAGE %%%%%%%%%%%%%%%%%%%

% Title of the paper, and the short title which is used in the headers.
% Keep the title short and informative.
\title[The transition redshift as an EoR probe]{The transition redshift as a probe of reionization}

% The list of authors, and the short list which is used in the headers.
% If you need two or more lines of authors, add an extra line using \newauthor
\author[A. Beane and K. Moriwaki]{
Angus Beane$^{1}$\thanks{E-mail: angus.beane@cfa.harvard.edu}
and Kana Moriwaki$^{2}$\thanks{E-mail: kana.moriwaki@utap.phy.s.u-tokyo.ac.jp}
\\
% List of institutions
$^{1}$Center for Astrophysics {\normalfont |} Harvard \& Smithsonian, 60 Garden Street, Cambridge, MA 02138, USA\\
$^{2}$Department of Physics, The University of Tokyo, 7-3-1 Hongo, Bunkyo, Tokyo 113-0033, Japan
% $^{2}$Center for Computational Astrophysics, Flatiron Institute, 162 5th Avenue, New York, NY 10010, USA\\
% $^{3}$Department of Physics \& Astronomy, University of Pennsylvania, 209 South 33rd Street, Philadelphia, PA 19104, USA
}

% These dates will be filled out by the publisher
\date{Accepted XXX. Received YYY; in original form ZZZ}

% Enter the current year, for the copyright statements etc.
\pubyear{2019}

% Don't change these lines
\begin{document}
\label{firstpage}
\pagerange{\pageref{firstpage}--\pageref{lastpage}}
\maketitle

% Abstract of the paper
\begin{abstract}
The early stages of the Epoch of Reionization (EoR), probed by the 21~cm line,
are sensitive to the detailed formation of the first galaxies. An important
transition occurs on large scales at $\avg{x_i}\sim0.15$, when the 21~cm field
transitions from being positively correlated with the galaxy field to being
negatively correlated. The redshift at which this transition occurs (the
``transition redshift'') is, in principle, sensitive to both the structure of
the 21~cm field but also the galaxy field. However, by considering a fixed
model for the EoR and varying the parameters of a simple model for the galaxy
field in a mock [O~\textsc{iii}]-emitter sample cross-correlation, we show
that the transition redshift is in fact insensitive to the structure of the
selected galaxy sample. We show that prospects for measuring the transition
redshift are feasible in the near-term with a combination of LOFAR and JWST.
Being derived from a cross-correlation, the transition redshift is less
impacted by foreground contamination currently inhibiting 21~cm auto-spectrum
measurements. Therefore, the transition redshift is an interesting probe of
the early stages of the EoR, and its detectability and interpretability
deserve further study.
\end{abstract}

% Select between one and six entries from the list of approved keywords.
% Don't make up new ones.
\begin{keywords}
cosmology: theory -- dark ages, reionization, first stars -- large-scale
structure of Universe
\end{keywords}

%%%%%%%%%%%%%%%%%%%%%%%%%%%%%%%%%%%%%%%%%%%%%%%%%%

%%%%%%%%%%%%%%%%% BODY OF PAPER %%%%%%%%%%%%%%%%%%

\section{Introduction} \label{sec:intro}
The redshifted 21~cm line contains a great deal of information about the
structure and evolution of the Epoch of Reionization (EoR). Once measured at
$z\sim6\textup{--}12$, it should reveal the nature of the formation of the
first stars, galaxies, and black holes, as well as the large-scale structure
of the Universe at high redshift \citep{2013fgu..book.....L}. While
instruments have reached the sensitivity necessary to detect the signal during
the EoR, issues of foreground contamination and instrumental artifacts have so
far prevented the success of such efforts. Only upper limits have been placed
on the amplitude of the 21~cm fluctuations \citep[e.g.][]{2013MNRAS.433..639P,
2014PhRvD..89b3002D, 2016ApJ...833..102B, 2017ApJ...838...65P}.

Because of the great difficulties confronting 21~cm auto-spectrum
measurements, the prospect of measuring 21~cm in cross-correlation with
another tracer of large-scale structure, such as galaxies. The galaxy field
will only correlate with the cosmological 21~cm signal, and the foregrounds
will not contribute to the mean signal (though they will make the measurement
noisy, and so some foreground subtraction is still necessary).

The 21~cm-galaxy cross-correlation was first explored by
\citet{2007MNRAS.375.1034W} in the context of using Ly$\alpha$-emitters to
distinguish between ``inside-out'' and ``outside-in'' reionization scenarios.
However, this technique is limited to the late stages of reionization since
Ly$\alpha$-emitters are obscured by neutral hydrogen even slightly earlier
into the EoR \citep[e.g.][]{2006ApJ...648....7K}. \citet{2007ApJ...660.1030F}
expanded upon this by considering a general galaxy survey. It was later shown
that the scale at which the cross-correlation goes from negative to zero (the
``turnover scale'') provides a tracer of the bubble size. However, the precise
turnover scale also depends on the properties of the selected galaxy sample,
with turnover happening on larger scales if only more massive galaxies can be
detected \citep{2009ApJ...690..252L}. A number of authors have studied this
cross-correlation in the context of detectability of the anti-correlation on
large scales \citep{2013MNRAS.432.2615W, 2016MNRAS.457..666V,
2017ApJ...836..176H}, and its dependence on the details of the EoR
\citep{2014MNRAS.438.2474P}.

In addition to the cross-correlation between 21~cm and galaxies, the
cross-correlation between 21~cm and intensity maps has been investigated in an
analogous fashion. Line intensity mapping experiments aim to measure the
large-scale structure of emission in a certain line, and is easier to perform
over a large area at high redshift than a galaxy survey. Current targets
include [\CII], CO, Ly$\alpha$, and H$\alpha$ \citep[for a recent review,
see][]{2017arXiv170909066K}. Another intriguing possible line is [\OIII]
$88\,\mu \text{m}$ \citep[e.g.][]{2018MNRAS.481L..84M}, in which a lensed
$z=9.1$ galaxy has recently shown to be exceptionally bright
\citep{2018Natur.557..392H}.

This cross-correlation was first studied with the CO rotational lines
\citep{2011ApJ...741...70L}, and then the [\CII] \citep{2012ApJ...745...49G}
and Ly$\alpha$ lines \citep{2013ApJ...763..132S}. The possibility of using the
cross-bispectrum the 21~cm and [\CII] fields has been explored
\citep{2018ApJ...867...26B}, along with combining the three cross-spectra with
a 21~cm, [\CII], and [\OIII] overlap \citep{2019ApJ...874..133B}. While in
this work we will only consider the cross-correlation between 21~cm and
galaxies, all of what we consider is applicable to the transition redshift as
computed by cross-correlating 21~cm with any of these emission lines.

In the case of either galaxies or emission lines, it is expected that the two
are positively correlated at the very beginning of the EoR (though the
emission lines may be very weak due to the lack of significant metal
enhancement). This is because the densest regions host the most mass and
therefore 21~cm emission and galaxy populations. However, the densest regions
will also ionize first {\bf cite something}, and after some time 21~cm and
galaxies will become negatively correlated. We refer to the redshift at which
this transition occurs at a fixed $k$ as the ``transition redshift''
$z_\star(k)$, assuming that only one such zero exists. In other words, the
cross-correlation at $k$ vanishes at $z=z_{\star}(k)$.

In this work we investigate to what extent the transition redshift is
dependent upon the galaxy sample used to define it. We assume a galaxy catalog
selected using an [\OIII] emitter survey. In particular, we are interested in
how $z_\star$ changes as we vary the parameters of the [\OIII] luminosity-mass
relation for a fixed halo catalog.

\section{Methods} \label{sec:methods}
\subsection{The transition redshift} \label{ssec:ztran_def}
The transition redshift $\zst(k)$ is defined by a field $x$ which traces the
density field on large scales and the condition that
\beq \label{eq:ztran_def}
P_{21,x}(k, \zst(k)) \equiv 0\text{.}
\eeq
If more than one such \zst{} exists, then we choose the lowest one where
$P_{21,x}$ last transitioned from positive to negative.

\subsection{21cmFAST} \label{ssec:tfast}
We generate realizations of the 21~cm field for our different EoR models using
the publicly available code \tfast{}
v1.3\footnote{\url{https://github.com/andreimesinger/21cmFAST}, commit:
\texttt{42e7566}}. Spin temperature fluctuations are computed and the
excursion set approach is used to identify the ionized regions
\citep{2011MNRAS.411..955M}. Rather than using the mean collapsed fraction, we
use the halo field to compute the ionization and 21~cm fields, since we are
interested in the evolution at early redshifts where the Poisson statistics of
the halo field can be important.

The parameters used in our \tfast{} runs are as folows. The minimum halo mass
contributing to the ionizing budget is $10^{8}\,\Msun$. The \HII{} efficiency
factor ($\zeta$) is set to $10$. The maximum smoothing scale for the
ionization field $R_{\text{max}}$ (loosely referred to as the ``mean-free
path'') is set to $30\,\Mpc$, though this quantity has almost no effect on the
early stage of the EoR. We set the fraction of baryons converted to stars
($f_\star$) to be $0.1$. We ignore redshift-space distortions for convenience.
All other parameters are set to their default values, except for the heating
parameter discussed in the next paragraph. The exact parameter files are given
in the github repository associated with this work (see the Supplementary
Material).

Our three different models for the EoR differ in the efficiency of their
heating, controlled by the number of X-ray photons per solar mass in galaxies
($\zeta_X$). Our fiducial model (\fid{}) is the default value
$\zeta_X=2\times10^{56}$, while our ``hot'' and ``cold'' models (\hot{} and
\cold{}) set $\zeta_X=10^{57}$ and $5\times10^{55}$, respectively. The average
$\delta T_b$ signal is shown in \textbf{some figure}, which shows
\textbf{something}. Note that none of these models account for the recent
claimed detection by the EDGES group of an extremely deep absorption trough at
$z\sim17$ \citep{2018Natur.555...67B}. If confirmed, though, that detection
would indicate an extremely low gas temperature at $z\sim17$, which would
favor larger amplitudes of the spin temperature fluctuations at $z\sim10$,
unless heating were extremely efficient after $z\sim17$. {\bf don't know how I
feel about the last sentence}



\subsection{[\OIII] luminosity} \label{ssec:oiii_lum}


In order to model the relation between [\OIII] luminosity ($L_{\OIII}$) and
halo mass, we follow \citet{2019ApJ...874..133B}. Specifically, we assume that
the luminosity for halo $j$
\beq \label{eq:lum_mass_relation}
L_j(M) = L_0 \left[ \frac{M}{M_0} \right]^{\alpha} + s_j(w)
\eeq
where $L_0/M_0^{\alpha}$ is set to match the total average intensity of
[\OIII] at that redshift, $\alpha$ is the power-law index, and $s_j(w)$ is a
number randomly selected from a lognormal distribution with zero mean and
width $w$. Once the average intensity is fixed (which we assume so), this is a
two-parameter model ($\alpha$ and $w$) of the [\OIII] luminosity-mass
relation.

\section{Detectability}


\section{To cite}
Papers that need to be cited:

The Cross-Correlation of High-Redshift 21 cm and Galaxy Surveys
\url{https://arxiv.org/abs/astro-ph/0611274}

Probing Reionization with the 21 cm-Galaxy Cross Power Spectrum
\url{https://arxiv.org/abs/0806.1055}

The HI Bias during the Epoch of Reionization, discusses the positive/negative
correlation transition as a function of $k$
\url{https://arxiv.org/abs/1907.05098}

\section{Conclusions} \label{sec:conclusions}

The conclusions.

\section*{Acknowledgements}

The Acknowledgements section.

%%%%%%%%%%%%%%%%%%%%%%%%%%%%%%%%%%%%%%%%%%%%%%%%%%

%%%%%%%%%%%%%%%%%%%% REFERENCES %%%%%%%%%%%%%%%%%%

% The best way to enter references is to use BibTeX:

\bibliographystyle{mnras}
\bibliography{references} % if your bibtex file is called example.bib

\appendix

\section{Some extra material}

The appendix.

%%%%%%%%%%%%%%%%%%%%%%%%%%%%%%%%%%%%%%%%%%%%%%%%%%


% Don't change these lines
\bsp	% typesetting comment
\label{lastpage}
\end{document}